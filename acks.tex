\chapter*{Предисловие и благодарности}

Это перевод на русский язык чернового текста книги, предназначенной для занятий
по операционным системам. В ней рассказываются основные концепции операционных
систем на примере ядра под названием xv6. Xv6 создано по образцу Unix 6 (v6)
Денниса Ричи и Кена Томпсона .(v6)~\cite{unix}. Ядро xv6 в общих чертах
повторяет структуру и стиль v6, но реализовано согласно стандарту ANSI
C~\cite{kernighan} для многоядерных RISC-V процессоров~\cite{kernighan}.

Эту книгу следует читать параллельно с исходным кодом ядра xv6, руководствуясь
подходом, вдохновленным комментарием Джона Лайонса к 6-му изданию
UNIX~\cite{lions}. Ресурсы по v6 и xv6, включая лабораторные задания с
использованием xv6, можно найти по адресу
\url{https://pdos.csail.mit.edu/6.1810}.

Эта книга использовалась в курсах по операционным системам 6.828 и 6.1810 в
Массачусетском технологическом институте (MIT). Мы благодарим преподавательский
состав, научных сотрудников и студентов этих курсов, которые прямо или косвенно
внесли свой вклад в развитие xv6. В частности, мы хотели бы поблагодарить
Адама Белая, Остина Клементса и Николая Зельдовича. Наконец, мы хотели бы
поблагодарить людей, приславших нам письма об ошибках в тексте или предложениях
по его улучшению: Абуталиба Агаева, Себастьяна Бема, brandb97, Антона Бурцева,
Рафаэля Карвальо, Тедж Чаджеда, Расита Эскичиоглу, Color Fuzzy, Войцеха Гака,
Джузеппе, Тао Гуо, Хайбо Хао, Наоки Хаямы, Криса Хендерсона, Роберта Хилдермана,
Эдена Хохбаума, Вольфганга Келлера, Генри Лайха, Джина Ли, Остина Лью,
Павана Маддамсетти, Яцека Масиуляниека, Майкла Макконвиля, m3hm00d,
miguelgvieira, Марка Моррисси, Мухаммеда Мурада, Гарри Пана, Гарри Портера,
Сиюань Цяна, Аскара Сафина, Салмана Шаха, Хуан Ша, Викрама Шеноя,
Адеодато Симо, Руслана Савченко, Павла Щурко, Уоррена Туми, tyfkda, tzerbib,
Вануша Васвани, Си Ванга и Чанга Вей, Сэма Уитлока, Люси Шоу Ян и Мэн Чжоу.

Если вы заметили ошибки или у вас есть предложения по улучшению, пожалуйста,
Франсу Каашуку и Роберту Моррису,
(kaashoekrtm@csail.mit.edu, rtm@csail.mit.edu), или автору перевода
(prototype3628800@yandex.ru).
