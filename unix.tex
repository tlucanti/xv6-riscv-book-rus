\chapter{Интерфейсы операционной системы}
\label{CH:UNIX}

Работа операционной системы
\emph{(англ. operation system)} заключается
в разделении компьютера между несколькими программами
и предоставлении более практичного набора сервисов,
чем это обеспечивает аппаратное обеспечение
\emph{(англ. hardware)} в одиночку.
Операционная система управляет и абстрагирует
аппаратное обеспечение низкого уровня.
Таким образом, например,
текстовый процессор не должен беспокоиться о том,
какой тип дискового оборудования используется.
Операционная система разделяет
аппаратное обеспечение между несколькими программами,
чтобы они работали (или казались работающими) одновременно.
Наконец, операционные системы предоставляют
контролируемые способы взаимодействия программ,
с помощью котоорых они могут
обмениваться данными и работать вместе.

Операционная система предоставляет сервисы
пользовательским программам через интерфейс.
\index{дизайн интерфейса}
\index{interface design}
И, оказывается, что разработка хорошего интерфейса
является сложной задачей.
С одной стороны, желательно,
чтобы интерфейс был простым и компактным,
потому что это облегчает правильную реализацию.
С другой стороны, соблазнительно
расширить функционал замудренными возможностями.
Хитрость в разрешении этого противоречия
заключается в разработке интерфейсов,
которые базируются на нескольких простых механизмах,
которые могут быть комбинированы для обеспечения
большой общности и абстракции.

Эта книга использует единственную операционную систему
для иллюстации концепций операционных систем.
Данная операционная система (xv6)
предоставляет базовые интерфейсы,
введеннные операционной системой Unix~\cite{unix}
Кена Томпсона и Денниса Ритчи,
а также имитирует внутреннее устройство Unix.
Unix обеспечивает компактный интерфейс,
механизмы которого удобно комбинируются,
предоставляя поразительную общность.
Этот интерфейс был настолько успешен,
что современные операционные системы -
BSD, Linux, macOS, Solaris и даже,
в меньшей степени, Microsoft Windows -
имеют похожие на Unix интерфейсы.
Понимание xv6 - это хороший старт для понимания
любой из этих систем и многих других.

Как показано на рисунке~\ref{fig:os},
xv6 имеет традиционную форму ядра \emph{(англ. kernel)} -
\index{ядро}
\index{kernel}
специальной программы, которая предоставляет
обслуживание работающим программам.
Каждая работающая программа называется процессом \emph{(англ. process)}.
\index{процесс}
\index{process}
У каждого процесса свои:
память, содержащую инструкции, данные и стек.
Инструкции реализуют вычисления программы.
Данные - это переменные,
над которыми выполняются вычисления.
Стек организует вызовы процедур программы.
На одном компьютере обычно работает много процессов,
но только одно ядро операционной системы.

Когда процессу необходимо обратиться к ядру,
он совершает системный вызов \emph{(англ. system call)}.
\index{системный вызов}
\index{system call}
Системный вызов - это одна из возможностей обратиться
к операционной системе через предоставленный интерфейс.
Системный вызов выглядит как вызов
подпрограммы или функции из системной библиотеки.
При использовании, он входит в ядро,
выполняет требуемую операцию и возвращается
в то место откуда был вызван.
Таким образом, процесс чередует исполнение
в пользовательском пространстве \emph{(англ. user space)}
и пространстве ядра \emph{(англ. kernel space)}.
\index{пространство пользователя}
\index{used space}
\index{пространство ядра}
\index{kernel space}

В последующих главах будет описано,
как ядро использует механизмы аппаратной защиты,
предоставляемые центральным процессором
\footnote{
  Этот термин в целом относится к аппаратному элементу,
  выполняющему вычисления.
  \index{ЦП}
  \index{CPU}
  В некоторых источниках (например, спецификации RISC-V)
  также используются слова
  процессор \emph{(англ. processor)},
  ядро \emph{(англ. code)},
  и сердце \emph{(англ. heart)} вместо ЦП.
}
\emph{(англ. central processor unit - CPU)},
чтобы запретить доступ процессов к памяти,
пренадлежащей другим процессам на компьютере.
Ядро работает с необходимыми аппаратными привилегиями
для реализации этих механизмов защиты.
Пользовательские же программы выполняются без таких привилегий.
Когда пользовательская программа
совершает системный вызов,
аппаратура повышает уровень привилегий
и начинает выполнение специальной программы
для подготовки переключения дальнейшего исполнения
в пространстве ядра.

\begin{figure}[t]
\center
\includegraphics[scale=0.5]{fig/os.pdf}
\caption{Ядро и два пользовательских процесса.}
\label{fig:os}
\end{figure}

Совокупность системных вызовов,
которые предоставляет ядро,
является интерфейсом,
который видят пользовательские программы.
Ядро xv6 предоставляет подмножество
возможностей и системных вызовов,
которые традиционно предоставляют ядра Unix.
На рисунке~\ref{fig:api}
перечислены все системные вызовы xv6.

Остальная часть этой главы описывает
сервисы xv6---процессы, память, дескрипторы файлов,
каналы и файловую систему---и иллюстрирует их
фрагментами кода и обсуждением того,
как оболочка \emph{(англ. shell)}---
\index{оболочка}
\index{shell}
пользовательский интерфейс командной строки UNIX, использует их.
Применение системных вызовов оболочкой демонстрирует,
насколько тщательно они были разработаны.

Оболочка - это обычная программа,
которая читает команды от пользователя и выполняет их.
Тот факт, что оболочка является
пользовательской программой и не является частью ядра,
иллюстрирует мощь интерфейса системных вызовов:
в оболочке нет ничего особенного по сравнению
с любыми другими пользовательскими процессами.
Это также означает, что оболочку легко заменить.
В результате в современных UNIX-системах
доступно множество оболочек на выбор,
каждая со своим пользовательским интерфейсом
и возможностями скриптового языка.
Оболочка xv6---это простая реализация
оригинальной оболочки UNIX Bourne.
Ее исходный код можно найти в файле \lineref{user/sh.c}.

%%
%% 	Processes and memory
%%
\section{Процессы и память}

Процессы в xv6 состоят из пользовательской памяти
(инструкций, данных и стека)
и состояния процесса, хранящегося внутри ядра
и недоступного для пользователя.
Ядро xv6 выполняет разделение времени между процессами:
оно автоматически переключает доступный процессор
на один из процессов, ожидающих выполнения.
Когда процесс не выполняется,
xv6 сохраняет регистры ЦП процесса,
восстанавливая их при следующем запуске процесса.
Каждому процессу соответствует уникальный идентификатор процесса
\emph{(англ. PID - process identifier)}
\index{идентификатор процесса}
\index{process identifier}
\index{PID}
На одном процессоре выполняется всегда только один процесс.
Другие процессы находятся в состоянии ожидания
до тех пор пока ядро не решит переключить исполнение на них.

\begin{longtable}[c]{l|l}
\caption{Long table caption.\label{long}}\\

\hline
Системный вызов & описание \\
\hline\hline
\endfirsthead

\hline
Системный вызов & описание \\
\hline
\endhead

\hline
\endfoot


int fork()                              & Создает новый процесс, родителю возвращает \\
                                        & идентификатор дочернего процесса. У ребенка \\
                                        & статус возврата - 0. \\
\hline
int exit(int status)                    & Завершает текущий процесс; статус возврата \\
                                        & считывается вызовом wait(). Функция не \\
                                        & возвращается. \\
\hline
int wait(int *status)                   & Ждет завершения дочернего процесса; статус \\
                                        & возврата ребенка помещается в *status. \\
                                        & Возвращает идентификатор процесса \\
                                        & завершившегося ребенка. \\
\hline
int kill(int pid)                       & Завершает процесс с идентификатором PID, \\
                                        & возвращает 0, или -1 при ошибке. \\
\hline
int getpid()                            & Возвращает идентификатор текущего процесса. \\
\hline
int sleep(int n)                        & Останавливает исполнение на n системных \\
                                        & тактов. \\
\hline
int exec(char *file, char *argv[])      & Загрузить исполняемый файл и запустить его с \\
                                        & аргументами. Функция возвращается только при \\
                                        & ошибке. \\
\hline
char *sbrk(int n)                       & Увеличивает размер пренадлежащей процессу \\
                                        & памяти на n байт. Возвращает старую границу \\
                                        & (которая теперь является началом нового \\
                                        & предоставленного участка памяти). \\
\hline
int open(char *file, int flags)         & Открывает файл с заданными флагами, \\
                                        & возвращает дескриптор открытого файла, или -1 \\
                                        & при ошибке. \\
\hline
int write(int fd, char *buf, int n)     & Записывает n байт из буфера в файловый \\
                                        & дескриптор fd. Возвращает количество \\
                                        & записанных байт, или -1 при ошибке. \\
\hline
int read(int fd, char *buf, int n)    & Считывает из файлового дескриптора fd в буфер \\
                                      & n байт. Возвращает количество записанных байт, \\
                                      & 0 если достигнут конец файла, или -1 \\
                                      & при ошибке. \\
\hline
int close(int fd)                     & Закрывает файл, и освобождает дескриптор fd. \\
\hline
int dup(int fd)                       & Возвращает новый дескриптор, ссылающийся на \\
                                      & то же место в файле что и fd.\\
\hline
int pipe(int p[])                     & Создает безымянный канал между двумя \\
                                      & дескрипторами, помещает созданные декскрипторы \\
                                      & на чтение/запись в p[0]/p[1]. \\
\hline
int chdir(char *dir)                  & Изменяет текущую рабочую директорию. \\
\hline
int mkdir(char *dir)                  & Создает новую директорию. \\
\hline
int mknod(char *file, int, int)       & Создает новый файл устройства. \\
\hline
int fstat(int fd, struct stat *st)    & Помещает информацию об открытом файле с \\
                                      & дескриптором fd в структуру по указателю st. \\
\hline
int stat(char *file, struct stat *st) & Помещает информацию о файле с именем file \\
                                      & в структуру по указателю st. \\
\hline
int link(char *file1, char *file2)    & Создает альтернативное имя file2 \\
                                      & (жесткую ссылку) для файла file1. \\
\hline
int unlink(char *file)                & Удаляет файл file. \\
\hline
\end{longtable}

Процесс может создать новый процесс,
используя системный вызов \emph{\indexcode{fork}}.
\indexcode{Fork} предоставляет новому процессу точную копию
памяти, инструкций, и данных вызывающего процесса.
\indexcode{Fork} возвращается как в исходном,
так и в новом процессах.
В исходном процессе \indexcode{fork} возвращает
идентификатор созданного процесса.
В созданном процессе \indexcode{fork} возвращает ноль.
Исходный и новый процессы обычно называются
родительским \emph{(англ. parent)}
и дочерним \emph{(англ. child)}.
\index{родительский процесс}
\index{parent process}
\index{дочерний процесс}
\index{child process}


For example, consider the following program fragment written in the C
programming language~\cite{kernighan}:
\begin{lstlisting}[]
int pid = fork();
if(pid > 0){
  printf("parent: child=%d\n", pid);
  pid = wait((int *) 0);
  printf("child %d is done\n", pid);
} else if(pid == 0){
  printf("child: exiting\n");
  exit(0);
} else {
  printf("fork error\n");
}
\end{lstlisting}
The
\indexcode{exit}
system call causes the calling process to stop executing and
to release resources such as memory and open files.
Exit takes an integer status argument,
conventionally 0 to indicate success and 1 to indicate failure.
The
\indexcode{wait}
system call returns the PID of an exited (or killed) child of the
current process and copies the exit status of the child to the address
passed to wait; if none of the caller's children
has exited,
\indexcode{wait}
waits for one to do so.
If the caller has no children, \lstinline{wait} immediately
returns -1.
If the parent doesn't care about the exit status of a child, it can
pass a 0 address to
\lstinline{wait}.

\iffalse
In the example, the output lines
\begin{lstlisting}[]
parent: child=1234
child: exiting
\end{lstlisting}
might come out in either order (or even intermixed), depending on whether the
parent or child gets to its
\indexcode{printf}
call first.
After the child exits, the parent's
\indexcode{wait}
returns, causing the parent to print
\begin{lstlisting}[]
parent: child 1234 is done
\end{lstlisting}
Although the child has the same memory contents as the parent initially, the
parent and child are executing with separate memory and separate registers:
changing a variable in one does not affect the other. For example, when the
return value of
\lstinline{wait}
is stored into
\lstinline{pid}
in the parent process,
it doesn't change the variable
\lstinline{pid}
in the child.  The value of
\lstinline{pid}
in the child will still be zero.

The
\indexcode{exec}
system call
replaces the calling process's memory with a new memory
image loaded from a file stored in the file system.
The file must have a particular format, which specifies which part of
the file holds instructions, which part is data, at which instruction
to start, etc. Xv6
uses the ELF format, which Chapter~\ref{CH:MEM} discusses in
more detail.
Usually the file is the result of compiling a program's source code.
When
\indexcode{exec}
succeeds, it does not return to the calling program;
instead, the instructions loaded from the file start
executing at the entry point declared in the ELF header.
\lstinline{exec}
takes two arguments: the name of the file containing the
executable and an array of string arguments.
For example:
\begin{lstlisting}[]
char *argv[3];

argv[0] = "echo";
argv[1] = "hello";
argv[2] = 0;
exec("/bin/echo", argv);
printf("exec error\n");
\end{lstlisting}
This fragment replaces the calling program with an instance
of the program
\lstinline{/bin/echo}
running with the argument list
\lstinline{echo}
\lstinline{hello}.
Most programs ignore the first element of the argument array, which is
conventionally the name of the program.

The xv6 shell uses the above calls to run programs on behalf of
users. The main structure of the shell is simple; see
\lstinline{main}
\lineref{user/sh.c:/main/}.
The main loop reads a line of input from the user with
\indexcode{getcmd}.
Then it calls
\lstinline{fork},
which creates a copy of the shell process. The
parent calls
\lstinline{wait},
while the child runs the command.  For example, if the user
had typed
``\lstinline{echo hello}''
to the shell,
\lstinline{runcmd}
would have been called with
``\lstinline{echo hello}''
as the argument.
\lstinline{runcmd}
\lineref{user/sh.c:/runcmd/}
runs the actual command. For
``\lstinline{echo hello}'',
it would call
\lstinline{exec}
\lineref{user/sh.c:/exec.ecmd/}.
If
\lstinline{exec}
succeeds then the child will execute instructions from
\lstinline{echo}
instead of
\lstinline{runcmd}.
At some point
\lstinline{echo}
will call
\lstinline{exit},
which will cause the parent to return from
\lstinline{wait}
in
\lstinline{main}
\lineref{user/sh.c:/main/}.

You might wonder why
\indexcode{fork}
and
\indexcode{exec}
are not combined in a single call; we will see later that
the shell exploits the separation in its implementation of
I/O redirection.
To avoid the wastefulness of
creating a duplicate process and then immediately replacing it (with \lstinline{exec}),
operating kernels optimize the implementation of
\lstinline{fork}
for this use case by using virtual memory techniques such as
copy-on-write (see Section~\ref{sec:pagefaults}).

Xv6 allocates most user-space memory
implicitly:
\indexcode{fork}
allocates the memory required for the child's copy of the
parent's memory, and
\indexcode{exec}
allocates enough memory to hold the executable file.
A process that needs more memory at run-time (perhaps for
\indexcode{malloc})
can call
\lstinline{sbrk(n)}
to grow its data memory by
\lstinline{n}
bytes;
\indexcode{sbrk}
returns the location of the new memory.

%%
%% 	I/O and File descriptors
%%
\section{I/O and File descriptors}

A
\indextext{file descriptor}
is a small integer representing a kernel-managed object
that a process may read from or write to.
A process may obtain a file descriptor by opening a file, directory,
or device, or by creating a pipe, or by duplicating an existing
descriptor.
For simplicity we'll often refer to the object a file descriptor
refers to as a ``file'';
the file descriptor interface abstracts away the differences between
files, pipes, and devices, making them all look like streams of bytes.
We'll refer to input and output as \indextext{I/O}.

Internally, the xv6 kernel uses the file descriptor
as an index into a per-process table,
so that every process has a private space of file descriptors
starting at zero.
By convention, a process reads from file descriptor 0 (standard input),
writes output to file descriptor 1 (standard output), and
writes error messages to file descriptor 2 (standard error).
As we will see, the shell exploits the convention to implement I/O redirection
and pipelines. The shell ensures that it always has three file descriptors
open
\lineref{user/sh.c:/open..console/},
which are by default file descriptors for the console.

The
\lstinline{read}
and
\lstinline{write}
system calls read bytes from and write bytes to
open files named by file descriptors.
The call
\lstinline{read(fd},
\lstinline{buf},
\lstinline{n)}
reads at most
\lstinline{n}
bytes from the file descriptor
\lstinline{fd},
copies them into
\lstinline{buf},
and returns the number of bytes read.
Each file descriptor that refers to a file
has an offset associated with it.
\lstinline{read}
reads data from the current file offset and then advances
that offset by the number of bytes read:
a subsequent
\lstinline{read}
will return the bytes following the ones returned by the first
\lstinline{read}.
When there are no more bytes to read,
\lstinline{read}
returns zero to indicate the end of the file.

The call
\lstinline{write(fd},
\lstinline{buf},
\lstinline{n)}
writes
\lstinline{n}
bytes from
\lstinline{buf}
to the file descriptor
\lstinline{fd}
and returns the number of bytes written.
Fewer than
\lstinline{n}
bytes are written only when an error occurs.
Like
\lstinline{read},
\lstinline{write}
writes data at the current file offset and then advances
that offset by the number of bytes written:
each
\lstinline{write}
picks up where the previous one left off.

The following program fragment (which forms the essence of the program
\lstinline{cat})
copies data from its standard input
to its standard output.  If an error occurs, it writes a message
to the standard error.
\begin{lstlisting}[]
char buf[512];
int n;

for(;;){
  n = read(0, buf, sizeof buf);
  if(n == 0)
    break;
  if(n < 0){
    fprintf(2, "read error\n");
    exit(1);
  }
  if(write(1, buf, n) != n){
    fprintf(2, "write error\n");
    exit(1);
  }
}
\end{lstlisting}
The important thing to note in the code fragment is that
\lstinline{cat}
doesn't know whether it is reading from a file, console, or a pipe.
Similarly
\lstinline{cat}
doesn't know whether it is printing to a console, a file, or whatever.
The use of file descriptors and the convention that file descriptor 0
is input and file descriptor 1 is output allows a simple
implementation
of
\lstinline{cat}.

The
\lstinline{close}
system call
releases a file descriptor, making it free for reuse by a future
\lstinline{open},
\lstinline{pipe},
or
\lstinline{dup}
system call (see below).
A newly allocated file descriptor
is always the lowest-numbered unused
descriptor of the current process.

File descriptors and
\indexcode{fork}
interact to make I/O redirection easy to implement.
\lstinline{fork}
copies the parent's file descriptor table along with its memory,
so that the child starts with exactly the same open files as the parent.
The system call
\indexcode{exec}
replaces the calling process's memory but preserves its file table.
This behavior allows the shell to
implement \indextext{I/O redirection} by forking,
re-opening chosen file descriptors in the child,
and then calling \lstinline{exec} to run the new program.
Here is a simplified version of the code a shell runs for the
command
\lstinline{cat}
\lstinline{<}
\lstinline{input.txt}:
\begin{lstlisting}[]
char *argv[2];

argv[0] = "cat";
argv[1] = 0;
if(fork() == 0) {
  close(0);
  open("input.txt", O_RDONLY);
  exec("cat", argv);
}
\end{lstlisting}
After the child closes file descriptor 0,
\lstinline{open}
is guaranteed to use that file descriptor
for the newly opened
\lstinline{input.txt}:
0 will be the smallest available file descriptor.
\lstinline{cat}
then executes with file descriptor 0 (standard input) referring to
\lstinline{input.txt}.
The parent process's file descriptors are not changed by this
sequence, since it modifies only the child's descriptors.

The code for I/O redirection in the xv6 shell works in exactly this way
\lineref{user/sh.c:/case.REDIR/}.
Recall that at this point in the code the shell has already forked the
child shell and that
\lstinline{runcmd}
will call
\lstinline{exec}
to load the new program.

The second argument to \lstinline{open} consists of a set of
flags, expressed as bits, that control what \lstinline{open}
does. The possible values are defined in the file control (fcntl) header
\linerefs{kernel/fcntl.h:/RDONLY/,/TRUNC/}:
\lstinline{O_RDONLY},
\lstinline{O_WRONLY},
\lstinline{O_RDWR},
\lstinline{O_CREATE}, and
\lstinline{O_TRUNC},
which instruct \lstinline{open} to
open the file for reading,
or for writing,
or for both reading and writing,
to create the file if it doesn't exist,
and to truncate the file to zero length.

Now it should be clear why it is helpful that
\lstinline{fork}
and
\lstinline{exec}
are separate calls: between the two, the shell has a chance
to redirect the child's I/O without disturbing the I/O setup of the main shell.
One could instead imagine a hypothetical combined
\lstinline{forkexec} system call,
but the options for doing I/O redirection with such a call
seem awkward.
The shell could modify its own I/O
setup before calling \lstinline{forkexec} (and then
un-do those modifications); or
\lstinline{forkexec} could take instructions for I/O
redirection as arguments;
or (least attractively) every program like \lstinline{cat} could
be taught to do its own I/O redirection.

Although
\lstinline{fork}
copies the file descriptor table, each underlying file offset is shared
between parent and child.
Consider this example:
\begin{lstlisting}[]
if(fork() == 0) {
  write(1, "hello ", 6);
  exit(0);
} else {
  wait(0);
  write(1, "world\n", 6);
}
\end{lstlisting}
At the end of this fragment, the file attached to file descriptor 1
will contain the data
\lstinline{hello}
\lstinline{world}.
The
\lstinline{write}
in the parent
(which, thanks to
\lstinline{wait},
runs only after the child is done)
picks up where the child's
\lstinline{write}
left off.
This behavior helps produce sequential output from sequences
of shell commands, like
\lstinline{(echo}
\lstinline{hello};
\lstinline{echo}
\lstinline{world)}
\lstinline{>output.txt}.

The
\lstinline{dup}
system call duplicates an existing file descriptor,
returning a new one that refers to the same underlying I/O object.
Both file descriptors share an offset, just as the file descriptors
duplicated by
\lstinline{fork}
do.
This is another way to write
\lstinline{hello}
\lstinline{world}
into a file:
\begin{lstlisting}[]
fd = dup(1);
write(1, "hello ", 6);
write(fd, "world\n", 6);
\end{lstlisting}

Two file descriptors share an offset if they were derived from
the same original file descriptor by a sequence of
\lstinline{fork}
and
\lstinline{dup}
calls.
Otherwise file descriptors do not share offsets, even if they
resulted from
\lstinline{open}
calls for the same file.
\lstinline{dup}
allows shells to implement commands like this:
\lstinline{ls}
\lstinline{existing-file}
\lstinline{non-existing-file}
\lstinline{>}
\lstinline{tmp1}
\lstinline{2>&1}.
The
\lstinline{2>&1}
tells the shell to give the command a file descriptor 2 that
is a duplicate of descriptor 1.
Both the name of the existing file and the error message for the
non-existing file will show up in the file
\lstinline{tmp1}.
The xv6 shell doesn't support I/O redirection for the error file
descriptor, but now you know how to implement it.

File descriptors are a powerful abstraction,
because they hide the details of what they are connected to:
a process writing to file descriptor 1 may be writing to a
file, to a device like the console, or to a pipe.
%%
%% 	Pipes
%%
\section{Pipes}

A
\indextext{pipe}
is a small kernel buffer exposed to processes as a pair of
file descriptors, one for reading and one for writing.
Writing data to one end of the pipe
makes that data available for reading from the other end of the pipe.
Pipes provide a way for processes to communicate.

The following example code runs the program
\lstinline{wc}
with standard input connected to
the read end of a pipe.
\begin{lstlisting}[]
int p[2];
char *argv[2];

argv[0] = "wc";
argv[1] = 0;

pipe(p);
if(fork() == 0) {
  close(0);
  dup(p[0]);
  close(p[0]);
  close(p[1]);
  exec("/bin/wc", argv);
} else {
  close(p[0]);
  write(p[1], "hello world\n", 12);
  close(p[1]);
}
\end{lstlisting}
The program calls
\lstinline{pipe},
which creates a new pipe and records the read and write
file descriptors in the array
\lstinline{p}.
After
\lstinline{fork},
both parent and child have file descriptors referring to the pipe.
The child calls \lstinline{close} and \lstinline{dup} to make file descriptor
zero refer to the read end of the pipe,
closes the file descriptors in
\lstinline{p},
and calls \lstinline{exec} to run
\lstinline{wc}.
When
\lstinline{wc}
reads from its standard input, it reads from the pipe.
The parent closes the read side of the pipe,
writes to the pipe,
and then closes the write side.

If no data is available, a
\lstinline{read}
on a pipe waits for either data to be written or for all
file descriptors referring to the write end to be closed;
in the latter case,
\lstinline{read}
will return 0, just as if the end of a data file had been reached.
The fact that
\lstinline{read}
blocks until it is impossible for new data to arrive
is one reason that it's important for the child to
close the write end of the pipe
before executing
\lstinline{wc}
above: if one of
\lstinline{wc} 's
file descriptors referred to the write end of the pipe,
\lstinline{wc}
would never see end-of-file.

The xv6 shell implements pipelines such as
\lstinline{grep fork sh.c | wc -l}
in a manner similar to the above code
\lineref{user/sh.c:/case.PIPE/}.
The child process creates a pipe to connect the left end of the pipeline
with the right end. Then it calls
\lstinline{fork}
and
\lstinline{runcmd}
for the left end of the pipeline
and
\lstinline{fork}
and
\lstinline{runcmd}
for the right end, and waits for both to finish.
The right end of the pipeline may be a command that itself includes a
pipe (e.g.,
\lstinline{a}
\lstinline{|}
\lstinline{b}
\lstinline{|}
\lstinline{c)},
which itself forks two new child processes (one for
\lstinline{b}
and one for
\lstinline{c}).
Thus, the shell may
create a tree of processes.  The leaves of this tree are commands and
the interior nodes are processes that wait until the left and right
children complete.

% In principle, one could have the interior nodes run the left end of a
% pipeline, but doing so correctly would complicate the
% implementation. Consider making just the following modification:
% change \lstinline{sh.c} to not fork for \lstinline{p->left} and run
% \lstinline{runcmd(p->left)} in the interior process. Then, for
% example, \lstinline{echo hi | wc} won't produce output, because when
% \lstinline{echo hi} exits in \lstinline{runcmd}, the interior process
% exits and never calls fork to run the right end of the pipe.  This
% incorrect behavior could be fixed by not calling \lstinline{exit} in
% \lstinline{runcmd} for interior processes, but this fix complicates
% the code: now \lstinline{runcmd} needs to know if it's in an interior
% process or not.  Complications also arise when not forking for
% \lstinline{runcmd(p->right)}.  For example, with just that
% modification, \lstinline{sleep 10 | echo hi} will immediately print
% ``hi' and a new prompt, instead of after 10 seconds; this happens because \lstinline{echo} runs
% immediately and exits, not waiting for \lstinline{sleep} to finish.
% Since the goal of the \lstinline{sh.c} is to be as simple as possible,
% it doesn't try to avoid creating interior processes.

Pipes may seem no more powerful than temporary files:
the pipeline
\begin{lstlisting}[]
echo hello world | wc
\end{lstlisting}
could be implemented without pipes as
\begin{lstlisting}[]
echo hello world >/tmp/xyz; wc </tmp/xyz
\end{lstlisting}
Pipes have at least three advantages over temporary files
in this situation.
First, pipes automatically clean themselves up;
with the file redirection, a shell would have to
be careful to remove
\lstinline{/tmp/xyz}
when done.
Second, pipes can pass arbitrarily long streams of
data, while file redirection requires enough free space
on disk to store all the data.
Third, pipes allow for parallel execution of pipeline stages,
while the file approach requires the first program to finish
before the second starts.
% Fourth, if you are implementing inter-process communication,
% pipes' blocking reads and writes are more efficient
% than the non-blocking semantics of files.
%%
%% 	File system
%%
\section{File system}

The xv6 file system provides data files,
which contain uninterpreted byte arrays,
and directories, which
contain named references to data files and other directories.
The directories form a tree, starting
at a special directory called the
\indextext{root}.
A
\indextext{path}
like
\lstinline{/a/b/c}
refers to the file or directory named
\lstinline{c}
inside the directory named
\lstinline{b}
inside the directory named
\lstinline{a}
in the root directory
\lstinline{/}.
Paths that don't begin with
\lstinline{/}
are evaluated relative to the calling process's
\indextext{current directory},
which can be changed with the
\lstinline{chdir}
system call.
Both these code fragments open the same file
(assuming all the directories involved exist):
\begin{lstlisting}[]
chdir("/a");
chdir("b");
open("c", O_RDONLY);

open("/a/b/c", O_RDONLY);
\end{lstlisting}
The first fragment changes the process's current directory to
\lstinline{/a/b};
the second neither refers to nor changes the process's current directory.

There are system calls to create new files and directories:
\lstinline{mkdir}
creates a new directory,
\lstinline{open}
with the
\lstinline{O_CREATE}
flag creates a new data file,
and
\lstinline{mknod}
creates a new device file.
This example illustrates all three:
\begin{lstlisting}[]
mkdir("/dir");
fd = open("/dir/file", O_CREATE|O_WRONLY);
close(fd);
mknod("/console", 1, 1);
\end{lstlisting}
\lstinline{mknod}
creates a special file that refers to a device.
Associated with a device file are
the major and minor device numbers
(the two arguments to
\lstinline{mknod}),
which uniquely identify a kernel device.
When a process later opens a device file, the kernel
diverts
\lstinline{read}
and
\lstinline{write}
system calls to the kernel device implementation
instead of passing them to the file system.

A file's name is distinct from the file itself;
the same underlying file, called an
\indextext{inode},
can have multiple names,
called
\indextext{links}.
Each link consists of an entry in a directory;
the entry contains a file name and a reference
to an inode.
An inode holds
\indextext{metadata}
about a file, including
its type (file or directory or device),
its length,
the location of the file's content on disk,
and the number of links to a file.

The
\lstinline{fstat}
system call
retrieves information from the inode that a
file descriptor refers to.
It fills in a
\lstinline{struct}
\lstinline{stat},
defined in
\lstinline{stat.h} \fileref{kernel/stat.h}
as:
\begin{lstlisting}[]
#define T_DIR     1   // Directory
#define T_FILE    2   // File
#define T_DEVICE  3   // Device

struct stat {
  int dev;     // File system's disk device
  uint ino;    // Inode number
  short type;  // Type of file
  short nlink; // Number of links to file
  uint64 size; // Size of file in bytes
};
\end{lstlisting}

The
\lstinline{link}
system call creates another file system name
referring to the same inode as an existing file.
This fragment creates a new file named both
\lstinline{a}
and
\lstinline{b}.
\begin{lstlisting}[]
open("a", O_CREATE|O_WRONLY);
link("a", "b");
\end{lstlisting}
Reading from or writing to
\lstinline{a}
is the same as reading from or writing to
\lstinline{b}.
Each inode is identified by a unique
\textit{inode}
\textit{number}.
After the code sequence above, it is possible
to determine that
\lstinline{a}
and
\lstinline{b}
refer to the same underlying contents by inspecting the
result of
\lstinline{fstat}:
both will return the same inode number
(\lstinline{ino}),
and the
\lstinline{nlink}
count will be set to 2.

The
\lstinline{unlink}
system call removes a name from the file system.
The file's inode and the disk space holding its content
are only freed when the file's link count is zero and
no file descriptors refer to it.
Thus adding
\begin{lstlisting}[]
unlink("a");
\end{lstlisting}
to the last code sequence leaves the inode
and file content accessible as
\lstinline{b}.
Furthermore,
\begin{lstlisting}[]
fd = open("/tmp/xyz", O_CREATE|O_RDWR);
unlink("/tmp/xyz");
\end{lstlisting}
is an idiomatic way to create a temporary inode
with no name
that will be cleaned up when the process closes
\lstinline{fd}
or exits.

Unix provides
file utilities callable from the shell
as user-level programs, for example
\lstinline{mkdir},
\lstinline{ln},
and
\lstinline{rm}.
This design allows anyone to extend the command-line interface
by adding new user-level programs.  In hindsight this plan seems obvious,
but other systems designed at the time of Unix often built such commands into
the shell (and built the shell into the kernel).

One exception is
\lstinline{cd},
which is built into the shell
\lineref{user/sh.c:/if.buf.0..==..c./}.
\lstinline{cd}
must change the current working directory of the
shell itself.  If
\lstinline{cd}
were run as a regular command, then the shell would fork a child
process, the child process would run
\lstinline{cd},
and
\lstinline{cd}
would change the
\textit{child} 's
working directory.  The parent's (i.e.,
the shell's) working directory would not change.
%%
%% 	Real world
%%
\section{Real world}

Unix's combination of ``standard'' file
descriptors, pipes, and convenient shell syntax for
operations on them was a major advance in writing
general-purpose reusable programs.
The idea sparked a culture of ``software tools'' that was
responsible for much of Unix's power and popularity,
and the shell was the first so-called ``scripting language.''
The Unix system call interface persists today in systems like
BSD, Linux, and macOS.

The Unix system call interface has been standardized through the Portable
Operating System Interface (POSIX) standard.
Xv6 is
\textit{not}
POSIX compliant:  it is missing many system calls (including basic ones such as
\lstinline{lseek}),
and many of the system calls it does provide differ from the standard.
Our main goals for xv6 are
simplicity and clarity while providing a simple UNIX-like system-call interface.
Several people have extended xv6 with a few more system calls and a simple
C library in order to run basic Unix programs.  Modern kernels, however,
provide many more system calls, and many more kinds of kernel services, than
xv6.  For example, they support networking, windowing systems, user-level threads,
drivers for many devices, and so on.  Modern kernels evolve continuously and
rapidly, and offer many features beyond POSIX.

Unix unified access to multiple types of resources (files,
directories, and devices) with a single set of
file-name and file-descriptor interfaces.
This idea can be extended to more kinds of resources;
a good example is Plan 9~\cite{Presotto91plan9},
which applied the ``resources are files''
concept to
networks, graphics, and more.
However, most Unix-derived operating systems have
not followed this route.

The file system and file descriptors have been  powerful
abstractions.
Even so, there are other models for operating system interfaces.
Multics, a predecessor of Unix,
abstracted file storage in a way that made it look like memory,
producing a very different flavor of interface.
The complexity of the Multics design had a direct influence
on the designers of Unix, who aimed to build something simpler.
% XXX can we cut this, since its point is the same as the next paragraph?
% An operating system interface that went out of fashion
% decades ago but has recently returned is the idea of a virtual machine monitor.
% Such systems provide a superficially different interface from xv6,
% but the basic concepts are still the same:
% a virtual machine, like a process, consists of some memory and
% one or more register sets;
% the virtual machine has access to one large file called
% a virtual disk instead of a file system;
% virtual machines send messages to each other
% and the outside world using virtual network devices
% instead of pipes or files.

Xv6 does not provide a notion of users or of protecting
one user from another; in Unix terms, all xv6 processes
run as root.

This book examines how xv6 implements its Unix-like interface,
but the ideas and concepts apply to more than just Unix.
Any operating system must multiplex processes onto
the underlying hardware, isolate processes from each
other, and provide mechanisms for controlled
inter-process communication.
After studying xv6, you should be able to
look at other, more complex operating systems
and see the concepts underlying xv6 in those systems as well.

%%
\section{Exercises}
%%

\begin{enumerate}

\item Write a program that uses UNIX system calls to
``ping-pong'' a byte between two processes over a pair
of pipes, one for each direction. Measure the
program's performance, in exchanges per second.

\end{enumerate}

\fi